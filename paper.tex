% --------------------------------------------------------------------------
% Template for DCASE 2018 paper; to be used with:
%          dcase2018.sty  - DCASE 2018 LaTeX style file, and
%          IEEEbib.bst - IEEE bibliography style file.
% Adapted from spconf.sty and waspaa15.sty
% --------------------------------------------------------------------------

\documentclass{article}
\usepackage{dcase2018,amsmath,graphicx,url,times,booktabs, tabularx}

% Example definitions.
% --------------------
\def\defeqn{\stackrel{\triangle}{=}}
\newcommand{\symvec}[1]{{\mbox{\boldmath $#1$}}}
\newcommand{\symmat}[1]{{\mbox{\boldmath $#1$}}}

% Title.
% --------------------
\title{AUTHOR GUIDELINES FOR DCASE 2018 WORKSHOP MANUSCRIPTS}

% Single addresses (uncomment and modify for single-address case).
% --------------------
% \name{Author(s) Name(s)\thanks{Thanks to XYZ agency for funding.}}
% \address{Author Affiliation(s)}
%
% For example:
% ------------
% \address{School\\
%       Department\\
%       Address}

% Two addresses
% --------------------
%\twoauthors
%  {F\'elix Gontier and Mathieu Lagrange and Jean-Francois Petiot}
%    {LS2N, UMR6004 CNRS\\
%Ecole Centrale de Nantes, 2 Chemin de la Houssini\`ere\\
%     44322 Nantes, France \\
%     felix.gontier@ls2n.fr}
%  {Catherine Lavandier and Pierre Aumond}
%    {ETIS, UMR8051\\
%     Universit\'e Paris Seine, Universit\'e de Cergy-Pontoise\\
%     ENSEA, CNRS \\
%     95000 Cergy-Pontoise, France .}
     
\name{F\'elix Gontier$^{1}$,
       Mathieu Lagrange$^{1}$,
       Jean-Francois Petiot$^{1}$
       }
 \secondlinename{	  
       Catherine Lavandier$^{2}$,
       Pierre Aumond$^{3}$
       }
       % fixed *.sty to allow names on multiple lines
 \address{$^1$ LS2N, UMR 6004, Ecole Centrale de Nantes, CNRS, 44322 Nantes, France, \{felix.gontier\}@ls2n.fr\\          
         $^2$ ETIS, UMR 8051, Universit\'e Paris Seine, Universit\'e de Cergy-Pontoise, ENSEA, CNRS, \\ 95000 Cergy-Pontoise, France, \{catherine.lavandier\}@u-cergy.fr\\ 
         $^3$ UMRAE, Ifsttar, 44341 Bouguenais, France, 
         \{pierre.aumond\}@ifsttar.fr\\
  }

%ETIS, UMR 8051, Université Paris Seine, Université de Cergy-Pontoise, ENSEA, CNRS, 95000 Cergy-Pontoise, France
% Authors in two lines, use in case of many authors with many affiliations (uncomment and modify).
% --------------------
% \name{John Doe$^{1}\sthanks{Thanks to ABC agency for funding.}$,
%       Maria Ortega$^{1}\sthanks{Thanks to XYZ agency for funding.}$,
%       Maria Doe$^{2}$, 
%       John Ortega$^{2}$,
%       John Maria$^{3}$, 
%       }
% \secondlinename{	  
%       David Smith$^{2}$, 
%       Maria Smith$^{3}$,
%       }
%       % fixed *.sty to allow names on multiple lines
% \address{$^1$ Fictional University, Computer Science Dept., Gotham, USA, \{john, maria\}@fictional.edu\\          
%         $^2$ University of the Imagination, Computer Science Dept., New Chicago, USA, \\
%         \{maria, john, david\}@fictional.edu\\ 
%         $^3$ University of the Fantasy, Department of Electronics, Pittsburgh, USA, 
%         \{john, maria\}@fantasy.edu\\
%  }

\begin{document}

\ninept
\maketitle

\begin{sloppy}

\begin{abstract}
\end{abstract}

\begin{keywords}
\end{keywords}


\section{Introduction}
\label{sec:intro}

\begin{enumerate}
\item Motivation (Noise, IOT...)
\item Project
\item Related work
\end{enumerate}

\section{Characterization Task}
\label{sec:format}

\begin{enumerate}
\item Soundscape perception models: perceptual space, pleasantness models, related sound sources
\item Hypothesis: Possible prediction from physical measurements (time presence)
\item Formal proposal: dataset, metrics, evaluation procedure
\end{enumerate}


\section{Perceptual validation}
\label{sec:pagelimit}

\begin{enumerate}
\item Experimental protocol: Test detailed description, dataset
\item Results
\begin{enumerate}
\item Perceptual space validation (ACP...)
\item Prediction of perceived time presence from separated sources
\end{enumerate}
\end{enumerate}

\section{Discussion}
\label{sec:pagestyle}

\begin{enumerate}
\item Conclusion on feasability
\item Outcome for task description
\end{enumerate}

% -------------------------------------------------------------------------
% Either list references using the bibliography style file IEEEtran.bst
\bibliographystyle{IEEEtran}
\bibliography{refs}
%
% or list them by yourself
% \begin{thebibliography}{9}
% 
% \bibitem{dcase2016web}
%   \url{http://www.cs.tut.fi/sgn/arg/dcase2016/}.
%
% \bibitem{IEEEPDFSpec}
%   {PDF} specification for {IEEE} {X}plore$^{\textregistered}$,
%   \url{http://www.ieee.org/portal/cms_docs/pubs/confstandards/pdfs/IEEE-PDF-SpecV401.pdf}.
%
% \bibitem{PDFOpenSourceTools}
%   Creating high resolution {PDF} files for book production with 
%   open source tools, 
%   \url{http://www.grassbook.org/neteler/highres_pdf.html}.
%
% \bibitem{eWilliams1999}
% E. Williams, \emph{Fourier Acoustics: Sound Radiation and Nearfield Acoustic
%   Holography}. London, UK: Academic Press, 1999.
% 
% \bibitem{ieeecopyright}
%   \url{http://www.ieee.org/web/publications/rights/copyrightmain.html}.
%
% \bibitem{cJones2003}
% C. Jones, A. Smith, and E. Roberts, ``A sample paper in conference
%   proceedings,'' in \emph{Proc. IEEE ICASSP}, vol. II, 2003, pp. 803--806.
% 
% \bibitem{aSmith2000}
% A. Smith, C. Jones, and E. Roberts, ``A sample paper in journals,'' 
%   \emph{IEEE Trans. Signal Process.}, vol. 62, pp. 291--294, Jan. 2000.
% 
% \end{thebibliography}


\end{sloppy}
\end{document}
